\label{memory/main}

\paragraph{}
There is three kind of memory managment on Xen.
The first one is the historical paravirtualized memory managment where the
guest is aware of its own virtualization.
The second one came with the first hardware assistance for virtualization,
allowing to run unmodified OS as guest with good performances.
Finally, the third appeared when the hardware provided an efficient
virtualization support in the MMU.

\paragraph{}
Before to see what mechanisms are enabled to allow the guests data to coexist
without interference, we have to introduce some vocabulary over the kind of
memory in an hypervisor as Xen.
Because the memory can be seen from three levels of privilege, there is three
kind of memory address:
\begin{description}
\item[Machine address (mfn)] is the ``true'' address of some data, as provided
  by the memory controller.
  When you boot the computer and the BIOS (or the bootloader) jump to your
  code, that is the kind of address you can play with.
\item[Physical address (gfn)] is the address the guest OS considers as true.
  When Xen boot a DomU and jump to its code, that is the kind of address the
  OS sees.
\item[Virtual address (vfn)] is the address the application see.
  It is your normal, clean, mmap provided memory, the one you love to play
  with until you get a segmentation fault.
\end{description}

\paragraph{}
There also is another taxonomy of address introduced with the need of nested
virtualization (or how to run an hypervisor under another hypervisor).
In this case the kind of address is given by the level of virtualization,
prefixed by a ``l'':
\begin{description}
\item[l0] is the address understood by the CPU without any virtualization.
  It is the kind of addresses the ``master'' hypervisor worry about.
\item[l1] is the address seen by the first level virtualized entity, for
  instance, a second hypervisor.
\item[l2] \ldots
\end{description}


\section{Paravirtualized memory managment}
\label{memory/paravirt}
(not yet)

\section{Hardware virtualized memory managment}
\label{memory/hvm}
(not yet)

\section{Hardware assisted memory managment}
\label{memory/hap}

\paragraph{}
Before to enter deeply in the code, let us ask the question ``how does the
hardware assisted memory managment works ?''.
There is actually two kind of memory managment, each provided by the two main
constructors of x86 processors, Intel and AMD.
We will only see how it works for AMD but the principle should be the same for
Intel.

\paragraph{}
Without virtualization it only exists virtual addresses and machine ones.
So the OS maintain a page table which associate with each virtual address a
machine one (a page table for each task).
When the application which knows only virtual address access one, the hardware
starts by looking into its TLB (cache for virtual to machine translation).
If it does not find the appropriate translation, the hardware will walk over
the page table (its start is pointed by the special register CR3) until finding
the mapping.
If the mapping is not present, a hardware interruption is generated and the OS
have to add the mapping.

\paragraph{}
Now in the case of hardware virtualization, the guest OS has to believe this
is what happen, but for obvious safety reasons, the guest will only map
virtual addresses to physical ones, and Xen will then provide a mapping from
physical to machine address, just like in \ref{memory/hvm}.
The difference is the shadow page table is not walked by software but directly
by the CPU.
Right before to enter into a guest, the hypervisor execute the assembly
instruction \verb!VMRUN! which tell the computer it is now running under
virtualization.
So back to the normal paging scheme, the application access to a virtual
address for the first time.
\begin{itemize}
\item The MMU watch into its TLB to see if there is a virtual to machine
  mapping.
\item Because it does not find, the MMU will walk the page table, but not the
  one indicated by CR3.
\item Instead it will look into the gCR3 register which is presented to the
  guest as the true CR3.
\item The gCR3 contains a physical address, so the MMU will look again in its
  TLB to see if there is a physical to machine mapping.
\item Let us say there is not, so the MMU will look into the real CR3 register
  (we call it hCR3) the location of the page table for the current guest.
\item The MMU now find the machine address for the guest page table, and it
  update the TLB.
\item The MMU read the first level of the guest page table and find the
  (physical) address of the second level for the wanted virtual address and
  try to access it.
\item Once again, the MMU look into its TLB, will maybe walk the physical to
  machine page table and this for the four (in 64-bit mode) levels.
\item Finally, the MMU get to the last level of the gCR3 pointed page table
  and see there is no mapping for the requested virtual address, and so
  throw a page fault to the guest OS (the hardware can store the guest
  interruption vector and use it when \verb!VMRUN!).
\item The guest OS will walk the page table, hopefully, the physical to machine
  mapping should be in the TLB, choose a physical page frame, create the
  mapping and perform and \verb!IRET!.
\item The MMU can now walk again the gCR3 page table and find the virtual
  to physical mapping and now it want to perform the very last physical to
  machine mapping to finally allow the application to access this page.
\item The MMU walk the hCR3 pointed page table and find no mapping for this
  last physical address and so generate a \verb!VMEXIT! exception with the
  error code \verb!NPF! (Nested Page Fault).
\end{itemize}

\paragraph{}
And now the funny stuff begins, and we will start to fall into the Xen code,
starting with \verb!svm_exit_handler()!
(xen/arch/x86/hvm/svm/svm.c).
So this function starts by checking what is the error reason.
The one we are interrested in is \verb!VMEXIT_NPF!, in this case it
will call \verb!svm_do_nested_pgfault()!.

\paragraph{}
In \verb!svm_do_nested_pgfault()! (xen/arch/x86/hvm/svm/svm.c), the first call
(to \verb!hvm_hap_nested_page_fault()!) is a trap, because the ``nested'' in
the function name here stands for the nested virtualization.
This special case brings a lot of complexity and we will ignore it.
Much more interresting is the \verb!__get_gfn_type_access()! because this is
the function which will bring back a fresh new page.

\paragraph{}
The \verb!__get_gfn_type_access()! (xen/arch/x86/mm/p2m.c) is also called in
paravirtualized mode (\ref{memory/paravirt}) and so the beginning of the
function is useless for us.
The things are getting cool when we get to the virtual call
\verb!p2m->get_entry()! which is intended to bring back an \verb!mfn! using
the hardware related (Intel or AMD) method.
We are looking for the AMD implementation which is the function
\verb!p2m_gfn_to_mfn()!.

\paragraph{}
This last function (xen/arch/x86/mm/p2m-pt.c) makes a first assignation
\verb!mfn = pagetable_get_mfn()! to get the top level of the guest related,
hCR3 pointed page table.
It will then walk this page table, allocating the missing levels if needed
until the last level.
Actually, this happens only if the DomU is configured as a ``page on demand''
guest, otherwise, all the mapping is done at the guest creation (which is the
default).
Nevertheless, we will see the ``page on demand'' case since it is a superset of
the default policy.
Both the intermediate and the last level allocations go throught the same
function \verb!p2m_pod_demand_populate()!, the only difference is the size
of the required page is larger when the level is nearer of the top.

\paragraph{}
The function \verb!p2m_pod_demand_populate()! (xen/arch/x86/mm/p2m-pod.c) is
a convenient way to access to what is called the POD cache (for Page On
Demand).
The POD cache consist in a simply linked list of 2 Megabytes pages and another
list of 4 Kilobytes pages.
There also is a counter indicating how many 4 KB pages there is in the cache
(each 2 MB page is counted as 512 4 KB pages).
The only interresting calls made in this function are:
\begin{itemize}
\item \verb!p2m_pod_cache_get()! which remove a 2 MB or 4 MB page from the
  cache, updating the lists (but not the counter, which is modified here).
\item \verb!set_p2m_entry()! to associate the physical address (\verb!gfn!)
  to the obtained machine address (\verb!mfn!).
\item \verb!set_gpfn_from_mfn()! to associate back the machine address to
  the physical one.
\end{itemize}
All the remainding is housekeeping and extremely rare error handling (such as
no more memory).

\paragraph{}
So when a page fault (or similarily a nested page fault) occurs, Xen will find
a free page in the POD cache.
If it fails, an out of memory error process is started and eventually, the
DomU will crash.
The next question is ``where does the pages of the POD cache come from ?''.
To find the answer, it is necessary to check the DomU creation code, and it
start with the \verb!xc_hvm_build()! function (actually, we could start upper
in the call stack but we are not interested in the \verb!xc! command code
here).

\paragraph{}
So the \verb!xc_hvm_build()! (tools/libxc/xc\_hvm\_build\_x86.c) is intended to
build and set up a new DomU from a Xen configuration file (already parsed).
Its role is simply to perform some control over the configuration arguments,
load the kernel image and call the \verb!setup_guest()! function.

\paragraph{}
There is some serious stuff in this function
(tools/libxc/xc\_hvm\_build\_x86.c) like kernel elf parsing, kernel module
loading etc\ldots
There is then some memory related code to try to check if Xen has enought
memory to give to the new guest and finally comes the interresting part:
Earlier in this section we made the distinction between two kinds of memory
managment: ``page on demand'' and the default one.
If the first, the function \verb!xc_domain_set_pod_target()! is called,
otherwise it is not (we will describe it a little later).
In both cases, \verb!xc! calls \verb!xc_domain_populate_physmap()!
(the previous call to the similar function
\verb!xc_domain_populate_physmap_exact()! is a kind of fast path).

\paragraph{}
This function \verb!xc_domain_populate_physmap()! (tools/libxc/xc\_domain.c) is
a wrapper for a Xen hypercall where the operation is
\verb!__HYPERVISOR_memory_op! and the command is
\verb!XENMEM_populate_physmap!.

\paragraph{}
Back in xen, our hypervisor catch this hypercall (as specified in the entry
file xen/arch/x86/x86\_64/entry.S) with the function \verb!do_memory_op()!
(located in xen/common/memory.c) which is basically a giant switch which will
brings us to the case \verb!XENMEM_populate_physmap! Xen make some
housekeeping with the flags and then call \verb!populate_physmap()!

\paragraph{}
In case of ``page on demand'', the function \verb!populate_physmap()! does
nothing (at least we will admit it for simplification reasons).
Else it will call the function \verb!alloc_domheap_pages()!.
Actually, if we are in ``page on demand'' policy, this function will also be
called.
If we backtrack a little in the \verb!xc! code at the
\verb!xc_domain_set_pod_target()!.

\paragraph{}
In this function \verb!xc_domain_set_pod_target()! (tools/libxc/xc\_domain.c)
which is an alias for \verb!xc_domain_pod_target()! (tools/libxc/xc\_domain.c),
\verb!xc! perform another hypercall with the operation
\verb!__HYPERVISOR_memory_op! and the command \verb!XENMEM_set_pod_target!.

\paragraph{}
This will be catch by the giant switch of the well known \verb!do_memory_op()!
(xen/common/memory.c) which will transfert it to the arch dependent handler
\verb!arch_memory_op()! (xen/arch/x86/mm.c) which is another giant switch we
are interrested in the case \verb!XENMEM_set_pod_target! which will finally
call \verb!p2m_pod_set_mem_target()!.

\paragraph{}
After a little control over the given arguments,
\verb!p2m_pod_set_mem_target()! (xen/arch/x86/mm/p2m-pod.c) will pass them to
the next function named \verb!p2m_pod_set_cache_target()! (xen/arch/x86/mm/p2m-pod.c)
which will call our previously introduced \verb!alloc_domheap_pages()!.

\paragraph{}
Now it comes the memory strategy, almost, because \verb!alloc_domheap_pages()!
(xen/common/page\_alloc.c) still need to determine on what memory zone it
will allocate pages (machine ones).
Whatever the zone, \verb!alloc_heap_pages()! is finally called.
Please also note than whatever the memory policy, this function is called
multiple times (once for each 2 MB or 1 GB chunk of memory, depending on the
policy).

\paragraph{}
This time is the good one, \verb!alloc_heap_pages()! is the function which
decides where the pages are allocated.
The algorithm is fairly simple, given a numa node mask (all nodes by default),
select the node next to the previously allocating one.
Then for this node, try to find a zone accordingly to the given zone
constraints which has enought free space (if several ones, take the smallest).
If such a zone is found, return it, otherwise, try the next node in the
nodemask.
If no node in the nodemask succeed, try in the other ones.
If no node succeed, then fails by returning NULL.

\paragraph{}
For each allocated chunk, the given (machine) address is assigned to the
domain (with \verb!assig_pages()! in xen/common/page\_alloc.c) and depending
on the memory managment, added to the POD cache, or mapped to the hCR3 pointed
page table.

